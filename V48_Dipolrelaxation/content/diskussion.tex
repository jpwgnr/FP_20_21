\section{Diskussion}
\label{sec:Diskussion}

Der Literaturwert für die Aktivierungsenergie bei verschiedenen Heizraten lautet ${W_{Lit} = \SI{0.66}{\electronvolt}}$ \cite{lit}. Damit weicht der Wert, der über das Anlaufverfahren bestimmt wurde und bei einer Heizrate von \SI{1.824(9)}{\kelvin\per\minute} gemessen wurde, am wenigsten ab mit einer relativen Abweichung von \SI{9.24}{\percent}. Der über das Integralverfahren mit der zweiten Heizrate hat mit \SI{36.97}{\percent} die größte relative Abweichung zum Literaturwert.

Der Literaturwert $\tau_{0, Lit}$ beträgt \SI{4e-14}{\second} \cite{lit}. Damit hat, analog zu der Messung der Aktivierungsenergie der Wert, der über das Anlaufverfahren gemessene Wert mit der ersten/zweiten Heizrate die geringste relative Abweichung zu dem Wert mit \SI{25}{\percent}. 
Die größte Abweichung hat erneut der Wert der zweiten Heizrate, der über das Integralverfahren ermittelt wurde. 
Die relative Abweichung liegt da bei \SI{99.98}{\percent}. 
Die große Abweichung erklärt sich durch die exponentielle Fehlerfortpflanzung des Aktivierungsenergiewerts. Daher sind die Unsicherheiten bei den Relaxationszeiten sehr groß.

Da der Versuch grundsätzlich eine statistische Messung ist, liegen auf den Werten statistische Unsicherheiten. Systematische Fehler die z.B. auf den Werten der Heizrate liegen und auf unsere Messfähigkeiten zurückzuführen sind, sind mit einer relativen Abweichung von \SI{0.49}{\percent} für die erste Heizrate und \SI{0.72}{\percent} für die zweite Heizrate sehr gering.