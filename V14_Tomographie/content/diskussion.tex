\section{Diskussion}
\label{sec:Diskussion}
Das in Abbildung \ref{fig:tfig1} gezeigte Spektrum der $^{137}$Cs-Quelle zeigt den erwarteten Verlauf.
So lässt sich sowohl die Compton-Kante, also auch der Photopeak eindeutig erkennen.

Das Material des homogene Würfels 2 wird Eisen zugeordnet, da der gemessene Absorptionskoeffizient nur um $1\%$ vom Literaturwert abweicht.
Die Zuordnung des Würfel 3 war weniger eindeutig.
Hier wird das Material als Delrin identifiziert, wobei der gemessene Absorptionskoeffizient um $26\%$ vom Literaturwert abweicht.
Fehler bei der Bestimmung der Zusammensetzung des unbekannten Würfels lassen sich möglicherweise darauf zurückführen, dass das Material des Würfel 3 nicht richtig bestimmt wurde.

In Tabelle \ref{tab:ttab6} sind die relativen Abweichungen der gemessenen Absorptionskoeffizienten des Würfels 5 zu denen des ihnen nach Tabelle \ref{tab:ttab5} zugeordneten Materials zu sehen.
\begin{table} 
    \caption{Relative Abweichung der Absorptionskoeffizienten der Elementarwürfel des Würfels 5.}
    \label{tab:ttab6}
    \centering
    \begin{tabular}{c | r}
    \toprule
        {} & {$\frac{|\mu_i-\mu|}{\mu}$} \\
        \midrule
        $\mu_1$ & 30,0 \% \\
        $\mu_2$ & 160,0 \% \\
        $\mu_3$ & 35,0 \% \\
        $\mu_4$ & 1,8 \% \\
        $\mu_5$ & 7,9 \% \\
        $\mu_6$ & 5,8 \% \\
        $\mu_7$ & 28,7 \% \\
        $\mu_8$ & 37,9 \% \\
        $\mu_9$ & 3,8 \% \\   
    \end{tabular}
    \end{table}

Bei einigen Elementarwürfeln ist die eindeutige Zuordnung zu einem der Materialien Eisen und Delrin aufgrund der Absorptionskoeffizienten möglich.
So weichen die Absorptionskoeffizienten $\mu_4$ und $\mu_9$ weniger als $5\%$ von den zugeordneten Werten ab.
Eine große Abweichung ist bei $\mu_2$ zu sehen.
Für die restlichen Koeffizienten liegt die Abweichung weit unter $50\%$.
Eine Fehlerquelle, die zu diesen Abweichungen geführt haben könnte, ist die Ausdehnung des Strahls.
So kann es sein, dass der Strahl zum Teil auch in Nachbarblöcke eindringt und die gemessene Intensität somit verfälscht wird.
Zudem kann auch eine ungenaue Justierung ein Grund für die Abweichungen sein.
Da es nach Augenmaß nicht möglich ist den Strahl bei jeder Projektion im gleichen Winkel auf den Würfel treffen zu lassen, legt dieser unterschiedlich lange Strecken durch das Material zurück.




