\section{Theorie}

Die Tomographie ist ein bildgebendes Verfahren, das ein Objekt schichtweise darstellt. Strahlung, hier Gammastrahlung, durchdringt das Objekt und liefert durch die Eingangsintensität $I_0$, die gemessenen Intensitäten $N$ und die Wegstrecken $d$ die daraus resultierenden Abschwächungen mit dem Absorptionskoeffizienten $\mu$, die ein Eigenschaft ist, mit der sich das Material identifizieren lässt. 
Der Ausdruck, der sich dann ergibt lautet
\begin{equation*}
    N = I_0 e^{- \sum_i \mu_i d_i}.
\end{equation*}
Für $j$ Messungen ergibt sich dann die Gleichung 
\begin{equation*}
    \sum_i d_{i,j} \mu_i = \ln \left( \frac{I_0}{N_j} \right),
\end{equation*}
bzw. in Matrixschreibweise 
\begin{align*}
\mathbf{A} \cdot \mathbf{\mu} = \mathbf{I}.
\end{align*}
Der Vektor $\mathbf{I}$ entspricht dabei natürlich den Elementen $\ln \left( \frac{I_0}{N_j} \right)$.
Hier wurden $i$ Würfel in einem großen Würfel angeordnet. Dabei bezieht sich der Absorptionskoeffizient $\mu_i$ auf den $i$-ten Würfel. Die Längen $d_{i,j}$ ändern sich je nach Würfel ($i$), ändern sich aber so wie die resultierenden Intensitäten je nach Richtung $j$.

Die Inverse der Matrix $\mathbf{A}$ und der Vektor $\mathbf{I}$ passen lassen sich nicht aufeinander anwenden. Daher wird hier ein Trick benutzt und man erweitert die Gleichung auf beiden Seiten mit $\mathbf{A^T}$. Daraus ergibt sich der Ausdruck 
\begin{equation*}
    \mathbf{\mu} = \left( \mathbf{A^T} \mathbf{A} \right)^{-1} \mathbf{A^T} \mathbf{I}.
\end{equation*}

Die Varianzen der Absorptionskoeffizienten ergibt sich über die Varianzen der gemessenen Intensität $\sigma_{I_j}$ und die Matrix $\mathbf{C_{I}}$ hat eben diese auf der Diagonalen stehen.

Die resultierenden Varianzen der Absorptionskoeffizienten stehen dann auf der Diagonalen der Matrix $\mathbf{C_{\mu}}$, die sich nach 
\begin{equation*}
    \mathbf{C_{\mu}} = \left( \mathbf{A^T} \mathbf{A} \right)^{-1} \mathbf{A^T} \mathbf{C_{I}} \mathbf{A} \left( \mathbf{A^T} \mathbf{A} \right)^{-1}
\end{equation*}
ergibt.

Die Gammastrahlung wechselwirkt mit Materie auf drei Arten. Der Prozess, der hier am wichtigsten ist, ist der Photoeffekt, dann über den Compton-Effekt, der hier hauptsächlich als Untergrund gemessen wird und über den Prozess der Paarbildung, für den die Energie des Photons bei zwei mal der Masse des Elektrons liegen muss.  
Eine exemplarische Darstellung der Wirkungsquerschnitte dieser drei Prozesse ist in Abb. \ref{abb:wirkungsquerschnitt} dargestellt. 