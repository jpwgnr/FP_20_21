\section{Diskussion}

Die Messungen an der Wasserprobe ergaben für die Relaxationszeiten
\begin{align*}
    T_1 &= \SI{2.96(4)}{\second} \\
    T_2 &= \SI{1.83(8)}{\second} \\.
\end{align*}
Dieses Verhalten war zu erwarten, $T_1$ ist größer als $T_2$. 
Das Verhältnis aus beiden Zahlen beträgt in etwa $\num{1.62}$. 
In Referenz \cite{chang} ergibt sich ein Verhältnis von $\num{1.85}$, was einer Abweichung von \SI{12.57}{\percent} entspricht. 

Der Literaturwert $T_1=\SI{3.09(15)}{\second}$ weicht um $4,2\%$ von dem gemessenen Wert ab.
Der Literaturwert für $T_2$ beträgt \SI{1.52(9)}{\second}, was einer Abweichung von \SI{20.4}{\percent} entspricht. 
Somit scheint die größere Abweichung bei der Messung von $T_2$ entstanden zu sein. Das kann an den zu geringen Pulsdauern liegen und daran, dass der verwendete Magnet nicht so stark war.

Die Abweichung zu den Literaturwerten lassen sich unter anderem dadurch erklären, dass die Probe nicht fixiert wurde und es somit bei jeder Messung der Temperatur zu einer leichten Positionsänderung gekommen ist.
Außerdem wird in Referenz \cite{chang} angemerkt, dass es zu Ordnung in der Wasserprobe kommen kann, die dafür sorgt, dass die Relaxationszeiten reduziert werden. 

Die Diffusionskonstante wurde zu 
\begin{equation*}
    D = \SI{1.882(33)e-9}{\meter\squared\per\second}
    \end{equation*}
bestimmt. 

Der Literaturwert des Diffusionskoeffizienten in Referenz \cite{chang} liegt bei \SI{2.78(4)e-9}{\meter\squared\per\second}, was einer Abweichung von \SI{32.4}{\percent} entspricht. Dies könnte sowohl an den Zeiten liegen, aber auch an der nicht ganz exakten Methode der Gradienten-Bestimmung. Die Anzahl der Messpunkte um \num{0} herum, war relativ gering bei dem Output der Fourier-Analyse. 




