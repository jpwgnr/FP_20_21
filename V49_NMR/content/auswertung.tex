\section{Auswertung}

Vor jedem Versuchsteil wurden jeweils die Temperaturen gemessen. 
Sie lag Konstant bei einem Wert von \SI{23.6}{\degree}.

Im ersten Versuchsteil wurde die Relaxationszeit $T_1$ gemessen. 
Die Werte, die bei dieser Messung gemessen wurden sind in Tab. \ref{tab:t1} aufgetragen. 
In Abb. \ref{abb:t1} sind die Daten dargestellt worden und nach Formel \eqref{eq:t1} wurde ein Fit an die Daten gelegt.

Die Parameter, die sich aus dem Fit ergeben haben sind 

\begin{align*}
U_1 &= 3 \\
t_1 &= 5 \\.
\end{align*}

Im nächsten Versuchsteil wurden die Zeit $t_2$ gemessen. 
Die Daten sind in Tab. \ref{tab:t2} aufgetragen und in Abb. \ref{abb:geradet2} als lineare Funktion aufgetragen worden. 
Erneut wurde ein Fit mit der Funktion \eqref{eq:t2} durchgeführt. 
Die Parameter, die sich daraus ergeben sind 

\begin{align*}
    U_1 &= 3 \\
    t_1 &= 5 \\.
\end{align*}

Die letzte zu messende Größe ist die Diffusionskonstante $d$. 
Dafür muss der Gradient $g$ mithilfe einer Fourier-Transformation gemessen werden. 
Das dazugehörige Spektrum ist in Abb. \ref{abb:spektrum} zu sehen. 
Nach der Fourier-Transformation ergibt sich folgende Abb. \ref{abb:fourier}. Der Abstand der Nullstellen auf der $x$-Achse beträgt demnach 
\begin{align*}
\gamma = \SI{12}{\hertz}.
\end{align*} 

Somit ergibt sich nach \eqref{eq:gradient} ein Wert von 
\begin{equation*}
g = \SI{12}{\per\metre}.
\end{equation*}

Das letzte was zur Bestimmung der Diffusionskonstante benötigt wird sind die gemessenen Echohöhen, die sich mit einem Fit der Funktion \eqref{eq:echo} ergeben. Daraus wurde der Koeffizient $D$ zu 
\begin{equation*}
D = \SI{3}{\metre}
\end{equation*}
bestimmt.

