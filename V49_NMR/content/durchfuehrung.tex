\section{Durchführung}
Vor Begin der Messung wird die Apparatur justiert.
Hierzu werden folgende Startparameter verwendet:
\begin{addmargin}[10pt]{0pt}
\begin{description}
    \item[Frequenz] $F=\SI{21,7}{\MHz}$
    \item[Pulslänge] $A=\SI{2}{\micro\m}$
    \item[Anzahl der B-Pulse] $N=0$
    \item[Periode] $P=\SI{0,5}{\s}$
    \item[Shimps] $x=-1,0;\; y=-0,5;\; z=3,7;\; z^2=-2,4$    
\end{description}
\end{addmargin}
Zunächst wird die Frequenz so eingestellt, dass das auf dem Oszillographen sichtbare Signal keine Schwingungen mehr aufweist.
Anschließend wird die Phase variiert bis das Signal des Realteils maximal und das des Imaginärteils minimal ist.
Um den Einfluss der Diffusionsprozesse in den ersten Messungen zu minimieren wird die Feldhomogenität des Feldes maximiert, indem der Gradient variiert wird bis die Abklingzeit des Signals möglichst lang ist.
Die benötigten Impulslängen werden durch Maximierung ($90°$-Puls) und Minimierung ($180°$-Puls) des Signals ermittelt.
Zuletzt wird die Temperatur in der Spule mit einem Thermoelement gemessen.
Die ermittelten Parameter lauten:
\begin{addmargin}[10pt]{0pt}
\begin{description}
    \item[Frequenz] $F=\SI{21,71370}{\MHz}$
    \item[Phase] $\phi = -51°$ 
    \item[Pulslängen] $\Delta t_{90}=\SI{2,7}{\micro\m}; \; \Delta t_{180}=\SI{5,2}{\micro\m}$
    \item[Temperatur] $T=296,75$    
\end{description}
\end{addmargin}

Für die $T_1$-Messung wird $A$ zunächst auf die $180°$-Pulslänge und $B$ auf die $90°$-Pulslänge gestellt.
Die Anzahl der Pulse der Länge $B$ wird auf $N=1$ und die Periode auf $P=\SI{10}{\s}$ eingestellt.
Anschließend wird der Pulsabstand $\tau$ variiert und die Amplitude des Signals vermessen.
Sobald $\tau > \SI{1}{\s}$ ist wird die Periode auf $T=\tau+\SI{10}{\s}$ verlängert.

Zur Messung von $T_2$ werden die Pulslängen von $A$ und $B$ getauscht und die Anzahl der $B$-Pulse auf $N=100$ gestellt, sodass 100 $180°$-Pulse auf einen $90°$-Puls folgen.
Der Schalter \textit{MG} wird auf \textit{on} gestellt, damit der $90°$-Puls um $90°$ gegenüber dem $180°$-Puls verschoben ist.
Der Pulsabstand $\tau$ wird so gewählt, dass die Amplitude des $100.$ Echos etwa $\textstyle{\frac{1}{3}}$ der des ersten Echos entspricht.
Dann wird ein Bild aller Peaks mit dem Oszilloskop gespeichert.
Anschließend wird der \textit{MG}-Schalter auf \textit{off} gestellt und auch dieses Bild gespeichert.

Für die Messung der Diffusionskonstanten wird zunächst der Gradient in $z$-Richtung maximiert indem der z-Shimp auf seinen maximalen Wert gestellt wird.
Die Werte für $A$ und $B$ werden aus der vorherigen $T_2$-Messung beibehalten und die Wiederholungen des $B$-Pulses auf $N=1$ reduziert.
Dann wird $\tau$ stetig vergrößert und die Höhe des Echos vermessen, bis das Signal im Rauschen verschwindet.
Bei einen Wert von $\tau$, für welchen sich ein gut sichtbares Echo ergibt, wird das Signal von Real- als auch Imaginärteil gespeichert.