\section{Zielsetzung}

Die Funktionsweise eines Diodenlasers wird im folgenden Versuch untersucht. Mit dem Diodenlaser wird das Transmissionsspektrum einer Rubidium-Probe untersucht. 

\section{Theorie}

\subsection{Aufbau und Funktionsweise}
Laser ist ein Akronym für \textit{Light amplification by stimulated emission of radiation}.
Ein allgemeiner Laser ist darauf ausgelegt kohärentes, monochromatisches Licht hoher Intensität zu erzeugen. Dabei werden Elektronen in einem Medium durch eine Pumpenergie aus dem Grundzustand in ein höheres Niveau befördert. Damit der Laser anfängt zu lasen muss eine Besetzungsinversion erzeugt werden. Diese ist erreicht, wenn mehr Elektronen in einem höheren Niveau sind als in dem unteren. Da bei einem Zwei-Niveausystem beide Niveaus mit derselben Wahrscheinlichkeit besetzt sind, wird dort nie eine Besetzungsinversion entstehen, somit wird ein Niveausystem mit drei oder mehr Niveaus benötigt. Durch zwei Spiegel werden die emittierten Photonen reflektiert und erneut durch das Medium geschickt. Dabei erzeugen Sie mit jedem Mal weitere Photonen aufgrund der Besetzungsinversion. Dieser Vorgang wird induzierte Emission genannt. Zwischen den Spiegeln entstehen stehende Wellen, deren Wellenlänge von dem Medium und der Resonatorlänge abhängig sind. 
Einer der beiden Spiegel transmittiert einen kleinen Teil des Lichts, wodurch der Laserstrahl aus dem Resonator gelangen kann, sobald dieser genug verstärkt und stabil ist. 

Diodenlaser sind spezielle Laser, die aus einem Diodenchip und einem äußeren Gitter zusammengesetzt sind. 
Der Diodenchip besteht aus einer Halbleiter-Heterostruktur. Der Aufbau ist in Abb. \ref{abb:chip} dargestellt. Am oberen und unteren Ende sind jeweils Elektroden angebracht, die die Pumpenergie in Form von einem Anregungsstrom erbringen. Dadurch kommt es zur Rekombination eines Elektron-Loch-Paares in der aktiven Schicht. Die Energie, die durch die Rekombination, also in Abhängigkeit der Bandlücke entsteht, wird in Form von Photonen emittiert. Die Funktion des Resonators ist durch zwei planparallele Schichten gegeben. Alle anderen Grenzflächen sind rau um mögliche Oszillationen in anderen Schichten der Diode zu verhindern.  

Da die Form der Austrittsöffnung rechteckig ist, ist das emittierte Licht stark divergent. Durch eine Linse wird es kollimiert. Das Licht in diesem Experiment hat eine Wellenlänge zwischen \SI{775}{\nm} und \SI{780}{\nm} bei einer Leistung von \SI{70}{\mW}. Die Linienbreite ist aber noch sehr groß, also im Bereich von \SI{50}{\MHz}, und damit ungeeignet für die Untersuchung atomarer Gegebenheiten. Außerdem ist die Frequenzstabilität empfindlich gegenüber Rückstreuung des Lichts in die Diode. Ein äußerer Resonator kann bei diesen Effekten helfen.
Der Resonator wird durch die zuvor beschriebene Linse und ein Beugungsgitter realisiert. Das Gitter kann sowohl vertikal als auch horizontal gedreht werden. Es besteht aus \SI{1800}{\per\milli\metre}. Dabei werden \SI{85}{\percent} direkt reflektiert und nur \SI{15}{\percent} zurück in den Chip reflektiert, wodurch die Linienbreite auf weniger als \SI{1}{\MHz} zurückgeht. Das in die Diode reflektierte Licht hat trägt durch induzierte Emission zusätzlich zur Stabilität des Lasers bei. 

\subsection{Laserverstärkung}

In Abb. \ref{abb:verstärker} ist die Verstärkung der verschiedenen Komponenten in Abhängigkeit der Wellenlänge aufgetragen. Dabei besitzt das aktive Medium eine Bandlücke abhängig vom Material, woraus ein breiter Peak in der Wellenlängenverteilung entsteht. Dieser ist abhängig von der Materialtemperatur. Für eine Resonanz in der Rubidiumprobe muss die Temperatur so eingestellt werden, dass eine Wellenlänge von \SI{780}{\nm} erreicht wird. Durch den Anregungsstrom ändert sich die Ladungsträgerdichte im aktiven Medium, wodurch sich der Brechungsindex $n$ verändert, was ebenfalls die Wellenlänge beeinflusst. In Abb. \ref{abb:temp} und \ref{abb:current} sind die Abhängigkeiten der Wellenlänge von der Temperatur und die Abhängigkeit vom Anregungsstrom aufgetragen. Durch Einstellen des horizontalen Winkels des Beugungsgitters, lässt sich die Wellenlänge ebenfalls einstellen nach der Bragg-Bedingung
\begin{equation*}
    \lambda = 2d \sin(\theta),
\end{equation*}
wobei $d$ der Gitterkonstanten entspricht. Insgesamt lässt sich die Wellenlänge also mithilfe der Temperatur, des Anregungsstroms und des Winkels des Beugungsgitters einstellen.



\subsection{Rubidium Absorptionsspektrum}

Das vom Laser emittierte Licht mit einer Wellenlänge von \SI{780}{\nm} wird von der Rubidiumprobe absorbiert, wenn dieser der Übergangsenergie des Isotops entspricht. Die Übergänge sind in Abb. \ref{abb:Rb} dargestellt. Die Übergänge lassen sich dann durch die Variation der Spiegeleinstellungen sichtbar machen und sind als Senkungen der Transmission an verschiedenen Stellen zu erkennen. Dabei werden vier Peaks erwartet. Zwei vom $^85Rb$ Isotop und zwei vom $^87Rb$ Isotop. 