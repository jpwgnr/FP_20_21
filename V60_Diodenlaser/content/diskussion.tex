\section{Diskussion}
\label{sec:Diskussion}

Sämtliche untersuchten Phänomene konnten in diesem Versuch beobachtet werden, was auf eine erfolgreiche Justage der Apparatur schließen lässt.
Trotzdem war es zum Teil schwierig die gewünschten Effekte sichtbar zu machen, da die Apparatur extrem empfindlich ist und durch minimale Drehungen der Knöpfe sehr große Effekte auf den Strahl des Lasers entstanden sind.
So war die Aufnahme des Fluoreszenzlichts dadurch erschwert, dass dieses nur bei ganz genauer Einstellung zu sehen war und alleine durch Loslassen des TOP-Knopfes häufig wieder verschwunden ist.
Zudem war es im letzten Versuchsteil nicht möglich die 87a Intensitätsabsenkungen deutlich erkennbar zu machen.
Auch nach mehrfacher Neujustage war nur eine minimale Absenkung zu erkennen.
Jedoch kann das in Abbildung \ref{abb:afig8} gezeigte Fluoreszenzspektrum von Rubidium eindeutig als solches identifiziert werden, da die gleichen Proportionen wie in Abbildung \ref{abb:Rb} zu sehen sind.